\documentclass[a4paper]{article}
\usepackage[utf8]{inputenc}
\usepackage{amssymb}
\usepackage[ngerman]{babel}
\usepackage{hyperref}
\usepackage{enumitem}
\usepackage{listings}
\usepackage{esvect}
\usepackage{float}
\usepackage{graphicx}
\usepackage{xcolor}
\usepackage{todonotes}
\usepackage{pgfplots}
\usepackage{verbatim}
\pgfplotsset{compat=1.10}
\usepgfplotslibrary{fillbetween}
\usetikzlibrary{patterns}
\usepackage{mathtools}
\usepackage{centernot}

\hypersetup{
     colorlinks   = true,
     citecolor    = gray
}

\title{Einführung in die Neuroinformatik Blatt1}
\author{Dominik Authaler, Carolin Schindler und Marco Deuscher\\Gruppe Q}
\date{Mai 2019}

\begin{document}

\maketitle

\section{Lineares Neuronenmodell}
\subsection{Differentialgleichung}
Mit den gegebenen Vereinfachungen ergebn sich aus der gegeben DGL die folgenden
vereinfachten DGL
\begin{align*}
    \tau \dot{u}_1&=-u_1(t)+x_1(t)\\
    \tau\dot{u}_2&=-u_2(t)+c_{12}y_1(t)=-u_2(t)+c_{12}u_1(t)
\end{align*}

Die maximale Ausgabe des ersten Neurons ist 1, da es keine größeren Eingangswerte gibt und $x_1(t)$ der einzige externe Eingangswert ist.\\
Die maximale Ausgabe des zweiten Neurons wird durch $c_12$ bestimmt. Der maximale Ausgang ist dann $\operatorname{max}(c_{12}*y_1(t))=\operatorname{max}(c_{12}*u_1(t))=c_{12}$.

\subsection{Diskretisieren der Gleichungen}
\begin{equation*}
    \dot{u}_i(t)\approx\frac{u_i(t-\Delta t)-u_i(t)}{\Delta t} \Rightarrow u_i(t+\Delta t)\approx \Delta t \dot{u}_i(t)+u_i(t)
\end{equation*}

Durch einsetzen der Gleichungen von oben erhält man dann
\begin{align*}
    u_1(t+\Delta t)&=u_1(t)(1-\frac{\Delta t}{\tau})+\frac{\Delta t}{\tau}x_1(t)\\
    u_2(t+\Delta t)&=u_2(t)(1-\frac{\Delta t}{\tau})+\frac{\Delta t}{\tau}c_{12}u_1(t)
\end{align*}

Diese Gleichungen können jetzt einfach in Python implementiert werden.


\subsection{Interpretation der Ergebnisse}
\paragraph{(a)}
Die Funktionswerte fallen ab $t=15$ ab, da die Anregung durch $x_1(t)$ ab diesem Zeitpunkt beendet ist. Da es keine Weiteren Eingangssignale gibt, fällt das Potenzial am Neuron ab.\\
Die Biologische Motivation
\todo[inline]{biologische Motivation}

\paragraph{(b)}


\paragraph{(c)}
Für den Fall $\tau=0$ gilt
\begin{align*}
    \tau \dot{u}_j(t)&=-u_j(t)+x_j(t)+\sum_{i=1}^n c_{ij}y_i(t-d_{ij})\\
    \Rightarrow u_j(t)&=x_j(t)+\sum_{i=1}^n c_{ij}y_i(t-d_{ij})\\
\end{align*}
Es ergibt sich also eine instantane Änderung des Potenzials, wenn sich eine der Eingangsgrößen ändert.

\paragraph{(d)}
$c_{12}$ beschreibt die Stärke der Bindungs zwischen Neuron 1 und Neuron 2. Die Gewichtung ist ein Maß dafür, wie stark das Potenzial an Neuron 1 das an Neuron 2 beeinflusst.\\
In unserem gibt es keine weiteren Eingangsgrößen an Neuron 2 und $y_1(t)=u_1(t)$. 

\paragraph{(e)}
Der Graph von $u_2(t)$ wird um $d_{12}$ nach recht verschoben.

\end{document}
